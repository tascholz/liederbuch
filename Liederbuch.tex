\documentclass{book}
\usepackage{ifpdf}
\usepackage[chorded]{songs}
\usepackage[margin=1.5cm]{geometry}

\nosongnumbers
\songcolumns{1}
\renewcommand{\everychorus}{Chorus:  }
\renewcommand{\lyricfont}{\LARGE}
\renewcommand{\printchord}{\bf\large}
\setlength{\cbarwidth}{0pt}
\setlength\baselineadj{-\baselineskip}
\baselineadj=-13pt plus 1pt minus 0pt

\newindex{mainindex}{idxfile}

\begin{document}
\showindex[1]{Songs}{mainindex}

 \begin{songs}{mainindex}
  \beginsong{Das Rattenpack}[by={Tjark Pelzer}]
  \beginverse
    Wir \[Bm]tragen keine \[G]Lederschuh und auch keine \[Bm]Seide \\
    Das Haar bedeckt kein \[Em]schöner Hut, die \[Bm]Haut ziert \[F#]kein Ge\[Bm]schmeide \\ 
    Schauen die Leut' auf \[G]uns herab und fühl'n sich schick und \[Bm]fein
    schneiden wir ihnen den \[Em]Beutel ab und \[Bm]lassen \[F#]eitel \[Bm]eitel sein
  \endverse
  \beginchorus
  \[Bm]Jeder von uns trug die Schlinge schonmal im Genack
  \[Em]Doch wir sind noch guter Dinge, \[Bm]haben's \[F#]immer \[Bm]rausgeschafft 
  \[Bm]Rauchen, saufen, schnorren, raufen, uns gehört die Stadt
  \[Em]Seht nur wie die Herren laufen, \[Bm]Hey, hier \[F#]kommt das \[Bm]Rattenpack  
  \endchorus
  \beginverse
  Tischmanieren kenn' wir nicht, woll'n sie auch nicht lernen
  teurer Wein im Magen sticht und kann dich auch nicht wärmen
  Wir bestell'n beim lieben Wirt lieber auf dich einen Schnaps
  und schaust du noch ganz verwirrt, sind wir weg mit einem Satz
  Jeder von uns...
  \endverse
  \beginverse
  Betteln gehen ist uns fremd, wollen's Geld lieber erspielen
  zocken noch ums letzte Hemd, bis wir's auch verlieren
  Haben wir dann kein' Heller mehr, holen wir das Messer raus
  sagen artig: Danke sehr! Zieh das Hemd doch auch noch aus
  Jeder von uns...
  \endverse
  \endsong
  
  \beginsong{All Voll}[by={Friedhelm Schneidewind}]
  \beginverse
  \[Dm]Bist du voll, so \[C]lege dich nieder, \[Dm]steh' früh auf und \[Am]völle dich \[Dm]wieder
  Das \[Dm]ganze \[C]Jahr, den \[F]Abend \[Bb]und den \[Dm]\[Am]\[Dm]Morgen
  Wein und Bier aus Faß und Krug saufe aus in einem Zug
  Das ganze Jahr, den Abend und den Morgen  
  \endverse
  \beginchorus
  All \[Dm]\[C]\[Dm]\[D]\[Gm]\[C]\[Dm]\[F]\[Gm]\[C]\[Dm]voll...
  \endchorus
  \beginverse
  Riechst du aus dem Maul nit nach Rosen, willst aber dennoch die Mägdlein liebkosen
  Das ganze Jahr, den Abend und den Morgen
  Schläfst des Nachts du dann in der Gosse, Schmutz und Unrat dein Schlafgenosse
  Das ganze Jahr, den Abend und den Morgen
  All voll...
  \endverse
  \beginverse
  Wein und Bier aus einem Fass, saufe aus ohn' Unterlass
  Das ganze Jahr, den Abend und den Morgen
  Bist du voll, so lege dich nieder, steh früh auf und völle dich wieder
  Das ganze Jahr, den Abend und den Morgen
  All voll...
  \endverse
  \endsong
  \beginsong{Abends geh'n die Liebespaare}[by={flo}, sr={Hermann Hesse}]
  \beginverse
  \[Dm]Abends geh'n die \[C]Liebespaare \[Dm]langsam \[C]durch das \[Dm]Feld
  \[Dm]Frauen lösen \[C]ihre Haare, \[Dm]Händler \[C]zählen \[Dm]Geld
  \[F]Bürger lesen \[C]bang das Neueste \[Dm]in dem \[C]Abend\[Dm]blatt
  \[F]Kinder ballen \[C]kleine Fäuste, \[Dm]schlafen \[C]tief und \[Dm]satt
  \[Bb]Jeder tut das \[F]einzig Wahre, \[C]folgt erhab'ner \[Dm]Pflicht   
  \[Bb]Säugling, Bürger, \[F]Liebespaare \[C]und ich selber \[Dm]nicht  
  \endverse
  \beginchorus
  \[Bb]Lei la \[F]lei la \[C]lala \[Dm]lei la, \[Bb]La \[F]la \[C]lei\[Dm]der x2
  \endchorus
  \beginverse
  Doch, auch meiner Abendtaten, deren Sklav' ich bin
  kann der Weltgeist nicht entraten, sie auch haben Sinn
  Und so geh' ich auf und nieder, tanze innerlich
  Summe dumme Gassenlieder, lobe Gott und mich
  Trinke Wein und phantasiere, dass ich Pascha wär'
  Fühle Sorgen an der Niere lächle, trinke mehr
  Weiter, weiter...
  \endverse
  \beginverse
  Sage ja zu meinem Herzen, morgen geht es nicht
  spinne aus vergangenen Schmerzen spielend ein Gedicht
  Sehe Mond und Sterne kreisen, ahne ihren Sinn,
  fühle mich mit ihnen reisen, einerlei wohin
  Leider, leider...
  Weiter, weiter...
  \endverse
  \endsong
  
  \beginsong{Als Spielmann gebor'n!!!!!!}[by={Tjark Pelzer}]
  \beginverse
  Die \[Dm]Straße zu Füßen, der \[C]Himmel mein \[Dm]Dach
  die \[F]Laute mein einz'ger Beg\[A]leiter
  die \[Dm]Lust an der Reise im \[C]Herzen einfacht
  so \[Dm]ziehe ich weiter und \[A]weiter
  \[F]Zieht mich ein Ort fesselnd \[C]in seinen Bann
  ver\[Dm]weile ich gerne dort \[A]drei Tage lang 
  \[F]Doch schon am vierten Tag \[C]treibt mich der Sinn  
  mit \[Dm]Sehnsucht schon \[A]wieder da\[Dm]hin
  \endverse
  \beginchorus
  \[F]Frei ist mein Sinn, wie die \[Dm]Vögel im Wind
  hab die \[Gm]Seel' an die Ferne ver\[A]lor'n
  und \[Dm]sperrt man mich ein, möcht ich \[A]tot lieber sein
  bin als \[Dm]fahrender \[A]Spielmann ge\[Dm]bor'n
  \endchorus
  \beginverse
  Ich spiel für den Zecher, den Wirt und die Maid
  den einsamen Mann an dem Tresen
  Ich spiel für die Tänzerin im blauen Kleid
  Ich spiel für die Sonne, den Regen
  Ich spiele des nachts, wenn der Pöbel lauscht
  vom Wein und von der Musik berauscht
  Ich spiele für jeden auf unserer Welt 
  doch niemals nur für das Geld
  Frei ist mein Sinn...
  \endverse
  \beginverse
  Und fängt mich auch einmal ein Mägdlein ein
  mit ihren süßen Lippen
  so will ich ein Weilchen der ihre gern sein
  und mich an ihr entzücken
  Doch wenn der pfeifende Wind mich ruft
  und mich mit seinem Klang versucht
  sieht sie wenn sie dann am Morgen erwacht
  ich bin entfloh'n in der Nacht
  Frei ist mein Sinn...
  \endverse
  \endsong
  \beginsong{Annabelle Lee}[by={Sarah Jarosz}, sr={Egar Allan Poe}]
  \beginverse
  \[Am]Many a year ago in a kingdom by the sea
   there \[C]lived a maiden you may \[G]know
  by the name of Annabelle \[Am]Lee
  No \[C]other thought did trouble her mind
  but to love and be loved by \[Am]me
  \endverse
  \beginverse
  We were children both in this kingdom by the sea
  but we loved with a love that was more than love
  I and my Annabelle Lee
  With a love that the winged angels high
  Coveted her and me
  \endverse
  \beginverse
  This was the reason long ago in this kingdom by the sea
  a wind blew from a stormy cloud
  that took my Annabelle Lee
  Then her wicked brothers came
  to steal her away from me
  \endverse
  \beginverse
  They shut her up in a tomb below this kingdom by the sea
  but no maiden's grave could sever my soul
  from the love that she bore for me
  For the moon don't beam without a dream
  of my darling Annabelle Lee
  \endverse
  \beginverse
  For many years I've wandered through this kingdom by the sea
  I've laid myself beside the bones
  of my beautiful Annabelle Lee
  I'll make my bed near the rising tide
  in her tomb by the sounding sea
  \endverse
  \endsong
  \beginsong{Auf die Ebbe}[by={Versengold}]
  \beginverse
  \[Em]Fühlt sich dein Kopp wie \[D]Treibgut an im \[Em]großen Meer der \[D]Zeit
  Bist \[Em]du besorgt und \[D]zweifelst dran, dass \[C]dir das \[D]Glück ge\[Em]weiht
  \[Em]Hast du nur Nebel \[D]im Gehirn und \[Em]vor der Stirn ein \[G]Brett  
  und \[Em]statt ner schönen \[D]Hafendirn 'ne \[C]Seeschlang\[D]e im \[Em]Bett
  \endverse
  \beginchorus
  \[Em]Kopf hoch, mein Freund, schenk \[D]ein! Ja, \[Bm]ist dein Glas auch \[A]wieder leer
  Das \[Em]Leben will ge\[G]feiert \[D]sein! Trink auf den \[G]Norden und \[A]trink aufs Meer
  \[Em]Kopf hoch, mein Freund, nur \[D]Mut! Ja, \[Bm]ist die Kehle auch \[A]trocken dir
  Sei \[Em]unbesorgt und \[G]glaube \[F]mir  
  Auf die \[C]Ebbe! Auf die \[G]Ebbe! Auf die \[A]Ebbe... Folgt die \[Em]Flut
  \endchorus
  \beginverse
  Hast du 'ne Flaute in der Hos', wenn du in der Koje liegst
  Und ärger mit der Ollen bloß, weil du nach Hering riechst
  Hat deine Braut von dir genug, ein gute Laune Leck
  und sowieso kein Holz vorm Bug und Miesmuscheln am Heck
  Kopf hoch...
  
  Hast \[Em]du nen Knoten im Gedärm und \[C]ist dir reichlich \[D]schlecht
  Be\[Em]steht die Welt nur noch aus Lärm, hast \[C]du zuviel ge\[D]zecht
  Fühlst \[Em]du dich wie das letzte Wrack, so \[C]streich nicht gleich die \[D]Segel  
  \[Em]Triff dich auf nen guten Schnack und \[C]hisse deinen \[D]Pegel!

  Auf die Eb\[B C B C]be...
  Auf die Eb\[B C B C]be...
  Folgt die \[Em]Flut!
  \endverse
  \endsong
  \beginsong{Auf in den Wind}[by={Versengold}]
  \beginverse*
  {\nolyrics Intro: \[Em] \[C] \[A] \[Em] \[C] \[A]}
  \endverse
  \beginverse
  \[Em]Lang ist der Tag wohl vergangen, als unser \[F]Leben einst mit mir ver\[Em]sank,
  \[Em]Als mich die Wogen verschlangen, als mich das \[F]nasskalte Dunkel er\[Em]trank
  \[F#m]Lang ist es her, dass das durstige Meer mich ver\[Em]darb,  
  \[F#m]als es mich trank, von dir nahm und mich nie wieder \[Em]gab
  \endverse
  \beginverse*
  Das Fleisch, was die Fische mir ließen,
  fristet sein Dasein hier mit mir an Bord
  Wir klagen im Nebel, wir fließen,
  wir stämmen die Ruder im Totenakkord
  Mein Herz, es weint mir voll Sehnen, das mich zu die zieht
  Die See, oh, sie schmeckt mir nach Tränen, seit ich von dir schied  
  \endverse
  \beginchorus
  \[Em]\[C]\[A]Aaah... Kalt weht der Wind übers Nebel\[Em]meer
  \[Em]\[C]\[A] Flüstert von Tod und von \[Em]Wiederkehr
  \[Em]\[C]\[A] Was die See nimmt gibt sie \[Em]nie \[C]wieder \[A]her
  \endchorus
  \beginverse
  Ich kann und ich will nicht vergessen,
  trotz all des Grauens, das hier um mich bebt
  Und während die Krebse mich fressen
  zeigt mir die Sehnsucht, dass ich einst gelebt
  Auf in den Wind, in die Hatz, wir sind Geister der See
  und da, wo wir sind, ist kein Platz für mein Leid, so versteh'
  Kalt weht der Wind...
  \endverse
  \beginverse*
  \[Bm]Ade, mein Lieb, ade mein \[D]tröstend Schoß
  Mich rief die See, grausam und \[A]gnadenlos  
  Mein Lieb, oh weh! Auf in den \[Em]Wind
  \[Bm]Mein Lieb, ade, aus deinen \[D]Armen  
  reif mich die See und ich deinen \[A]Namen
  Mein Lieb, oh weh! Auf in den \[Em]Wind!
  \endverse
  \beginverse*
  \[Em]Lang ist es \[C]her, dass du fort von mir \[A]gingst
  -Als ich am Horizont schweigend verschwand
  Ich will es nicht glauben, auch wenn ich's versteh
  -Ich sah dich dir stehen am schwindenden Strand
  Ich höre im Wind, wie du weinst, wie du singst
  -Ich bin auf ewig aufs Meere gebannt
  Und meine Tränen, sie schmecken so wie die See
  \endverse
  \beginverse*
  Ade, mein Lieb, ade mein tröstend Schoß
  mich rief die See, grausam und gnadenlos
  mein Lieb, o weh! Auf in den Wind!
  Mein Lieb, ade, aus deinen Armen
  rief mich die See und ich deinen Namen
  Mein Lieb oh weh! Auf in den Wind!
  Ich ruf deinen \[A]Namen - Auf in den \[Em]Wind
  Kalt weht der Wind...  
  \endverse
  \endsong
  \beginsong{Aurelius}[by={Tjark Pelzer}]
  \beginverse
  In \[Am]uns'rer Mitte lebt ein Held, Au\[F]relius \[E]war sein \[Am]Name
  Ein \[Am]Mühlrad dient als Armreif ihm, so \[F]stark sind \[E]seine \[Am]Arme 
  Die \[F]Faust so hart wie ein \[C]Schmiedehammer, der \[G]Blick so scharf wie ein \[C]Beil
  und mit \[Am]einem Streich von seinem Schwert haut er \[F]einen \[E]Amboss ent\[Am]zwei
  \endverse
  \beginverse
  Sein Herz schlägt stark wie zwanzig Mann, sein Mut ist unvergleichbar
  wo tapf're Männer zögernd stehn, da geht Aurelius weiter
  Nichts vermag ihn zu Fall zu bringen, nicht Frau, nicht Mann, nicht Tier
  und wer ihm gegenübersteht, dessen Arm vor Furcht gefriert
  \endverse
  \beginverse
  Aurelius weckt bei Frauen schier unmenschliches Verlangen
  Liegt er bei einem Weib so hat sie danach rote Wangen
  Im ganzen Land ist er bekannt für seine Manneskraft
  und für die Lust die seine Finger einem Weib verschafft
  \endverse
  \beginverse
  Und kommt der Tag an dem Aurelius geht zu seinen Ahnen
  ertrinkt die Welt im Tränenmeer der Reichen und der Armen
  Doch aus den Wolken hört man eine Stimme, die da sprach
  Ach, grämt euch nicht, ich gebe von hier oben auf euch acht
  \endverse
  \endsong
  \beginsong{Aussatz}[by={Cultus Ferox}]
  \beginverse
  Wir \[Am]gleiten auf den \[E]Planken Richtung \[F]schwarzen Pulver\[G]rausch
  Die \[Am]vollen Schiffe \[E]wanken, doch \[F]wir sind noch wohl \[G]auf
  Wir \[C]steigen in die \[G]Wanten und \[F]warten auf den \[G]Wind
  Mit \[C]einem Fass voll \[G]Rum wird ein \[F]neuer Kurs be\[G]stimmt
  \endverse
  \beginchorus
  Auf den \[F]Meeren sind wir \[C]ewig, \[G]frei und unbeug\[Am]sam
  sind wir \[F]Kaiser, Papst und \[C]König, aber \[G]niemals Bettel\[Am]mann
  \endchorus
  \beginverse
  Um dem Galgen zu entfliehen suchen wir die neue Welt
  Wo wir nicht den Heuchlern dienen, wo der alte Kodex zählt
  An Schnee und Eis vorbei und den größten Ungetümen
  Treiben uns die Stürme zu den Inseln tief im Süden
  Auf den Meeren...
  \endverse  
  \beginverse
  Dort schließen sich dem Bund noch viele Schiffe an
  Und auf den Meeresgrunde sinkt der Schuldspruch lebenslang
  Niemand weiß wohin uns're weite Reise geht
  Wir hissen stolz die Fahne auf der uns're Botschaft steht
  Auf den Meeren...
  \endverse
  \beginverse
  Freiheit sei beschieden, ob arm, ob reich, ob blind
  Dem Aussatz, den Vertrieb'nen, die alle mit uns sind
  Sie steigen in die Wanten und warten auf den Wind
  Mit einem Fass voll Rum wird ein neuer Kurs bestimmt
  Auf den Meeren...
  \endverse
  \endsong
  \beginsong{Ballade vom roten Haar}[by={Peter Rohland}, sr={Francois Villon}]
  \beginverse
  Im \[Am]Sommer war das \[Dm]Gras so tief, dass jeder \[G]Wind daran vorüber \[C]lief 
  Ich habe \[Am]da dein \[Em]Blut ge\[G]spürt und wie es \[F]heiß zu mir herrüber\[Am]rann
  Du \[Am]hast nur mein Ge\[Dm]sicht berrührt da starb er \[G]einfach hin, der harte \[C]Mann
  Weil's solche \[Am]Liebe \[Em]nicht mehr \[G]gibt, ich hab mich \[F]in dein rotes Haar ver\[F]liebt
  \endverse
  \beginverse
  Im Feld, den ganzen Sommer war der rote Mohn so rot nicht wie dein Haar
  Jetzt wird es abgemäht, das Gras, di bunten Blumen welken auch dahin
  Und wenn der rote Mohn so blass geworden ist, dann hat es keinen Sinn
  dess es noch weiße Wolken gibt, ich hab mich in dein rotes Haar verliebt
  \endverse
  \beginverse
  Du sagst, dass es bald Kinder gibt, wenn man sich in dein rotes Haar verliebt
  So rot wie Mohn, so weiß wie Schnee, im Herbst da kehren viele Wunder ein
  Warum sool's auch bei uns nicht sein, du bliebst im Winter acuh mein rotes Reh
  und wenn es tausend schön're gibt, ich hab mich in dein rotes Haar verliebt
  \endverse
  \endsong
  \beginsong{Barfuß geh'n}[by={Tjark Pelzer}]
  \beginverse
  In \[Cm]Aldradachs Schatten, in \[Gm]finsteren Gassen
  ent\[Ab]zogen dem \[Bb]suchenden \[Cm]Blick
  Ver\[Cm]bergen sich zwei, denen \[Gm]ist's einerlei, 
  ob dich \[Ab]Fürst oder \[Bb]Henker ge\[Cm]schickt
  Kein \[Ab]Betteln, kein Fleh'n, kein \[Cm]auf Knie gehen
  kein \[Bb]Drohen, kein freundliches \[Cm]Wort
  \[Cm]Bringt dich hier weiter, du \[Gm]bist längst gescheitert
  bist \[Ab]Fremder du \[Gm]an diesem \[Cm]Ort
  \endverse
  \beginchorus
  Drückt \[Cm]dir die Last des \[Gm]Lebens schwer, hast \[Ab]du zu\[Bb]viel Ge\[Cm]päck
  ist \[Ab]dir der Beutel \[Cm]allzu prall, wirf \[G]ihn nicht einfach \[Cm]weg
  Such \[Ab]nur die rechte \[Cm]Gasse auf, ver\[Bb]trau mir, du wirst \[Cm]seh'n
  das \[Ab]Leben wird dir \[Cm]leichter \[Fm]sein, auch \[Cm]du wirst \[G]barfuß \[Cm]geh'n
  \endchorus
  \beginverse
  Wenn dir auf den Magen so manch Dinge schlagen, 
  die dir deine Laune vergrell'n
  brauchst du nicht verzagen, denn ich kann dir sagen
  wo sich deine Sorgen einstell'n
  tritt ein in die Gassen, bekomm sie zu fassen
  die Zwiebeln der Zwei bringen dir Ruh'
  wer suchet der findet, der Hunger verschwindet,
  die Sorgen verschwinden dazu
  Drückt dir die Last...
  \endverse
  \beginverse
  Versuchst du mit Wörtern Dinge zu erörtern
  die diese zwei miskreditieren?
  Ein Haifisch kann beißen, den Hals dir zerreißen
  dein Ableben realisieren
  Kommt dieser nicht weiter, ist da noch ein Zweiter
  der glänzt durch beträchtlichen Charme
  sein Lächeln gewinnt, deine Vorsicht verrint
  und schon hast du zwei Löcher im Darm
  Drückt dir die Last...
  \endverse
  \endsong
  \beginsong{Birkenring}[by={Margarete und Wolfgang Jehn}, sr={traditionell}]
  \beginverse
  Warum \[Dm]zögerst du noch und bleibst \[A]steh'n in der Nacht
  Horch im Wald hinterm Dorf ist der \[Dm]Sommer erwacht
  Tritt doch näher mein Freund und reich \[A]mir deine Hand
  komm herein in den fröhlichen \[Dm]Birkenring 
  \endverse
  \beginchorus
  Sieh, das \[Gm]Glück wird vergeh'n, denn die \[Dm]Zeit, sie bleibt nicht steh'n
  Mit den \[A]Winden vom Meer wird der \[Dm]Sommer vergeh'n
  Drum drück \[Gm]fest an dein Herz, was die \[Dm]Freude dir gibt
  Komm her\[A]ein in den fröhlichen \[Dm]Birkenring
  \endchorus
  \beginverse
  Was die Kantele sagt, darfst du glauben, Freund
  Heut' wird wahr, was du einst im Winter geträumt
  Wenn die Liebe dir lacht, wend' nicht ab deinen Blick
  komm herein in den fröhlichen Birkenring
  Sieh das Glück wird vergeh'n...
  \endverse
  \beginverse
  Einmal kam ich auch schon als Wandersbursch her 
  Ich wollt' gern was erleben, ich wünschte es so sehr
  Und der Frühling war mir eine zweite Geburt
  komm herein in den fröhlichen Birkenring
  Sieh das Glück wird vergeh'n...
  \endverse
  \beginverse
  Und nun wohne ich hier im Wald bei dem Weib
  und im Winter da sagt mir die Kantele bald
  Wirst du staunen und schau'n, was du einsam geträumt
  Komm herein in den fröhlichen Birkenring
  Sieh das Glück wird vergeh'n
  \endverse
  \endsong
  \end{songs}
  
\end{document}
